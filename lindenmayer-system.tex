\documentclass[11pt]{report}
\usepackage[utf8]{inputenc}

\usepackage{xcolor}
\usepackage{minted}
\usepackage{caption}
\usepackage{fancyhdr}
\usepackage{lastpage}
\usepackage{titlesec}
\usepackage{hyperref}
\usepackage{tcolorbox}
\usepackage{pmboxdraw}

\usepackage[left=0.75in, right=0.75in, top=0.75in, bottom=1.5in]{geometry}

% bibliography management system using biblatex; backend: biber; citation style: IEEE
\usepackage[backend=biber,style=ieee]{biblatex}
\addbibresource{references.bib}

\renewbibmacro*{finentry}{
    \iffieldundef{annotation}
        {}
        {\par\smallskip\textit{Annotation:} \printfield{annotation}}
    \finentry
}

\author{Hollie-Mae Boittiaux}

\setlength{\headheight}{34pt}
\newcommand{\header}{\vspace{-10pt}\rule{\textwidth}{0.4pt}}
\setlength{\footskip}{33pt}

% pages which start on a chapter
\fancypagestyle{plain}{
    \fancyhf{}
    \fancyhead[L]{\header}
    \fancyfoot[C]{
    \makebox[\textwidth]{
      \rule[3pt]{0.425\textwidth}{0.6pt} 
      \hspace{4pt} 
      Page \thepage\ of \pageref{LastPage} 
      \hspace{4pt} 
      \rule[3pt]{0.425\textwidth}{0.6pt}
    }
  }
    \renewcommand{\headrulewidth}{0pt}
    \renewcommand{\footrulewidth}{0pt}
}

% pages which do not start on a chapter
\pagestyle{fancy}
\fancyhf{}
\fancyhead[L]{\header}
\fancyfoot[C]{
    \makebox[\textwidth]{
      \rule[3pt]{0.425\textwidth}{0.6pt} 
      \hspace{4pt} 
      Page \thepage\ of \pageref{LastPage} 
      \hspace{4pt} 
      \rule[3pt]{0.425\textwidth}{0.6pt}
    }
  }
\renewcommand{\headrulewidth}{0pt}
\renewcommand{\footrulewidth}{0pt}

\titleformat{\chapter}[hang]
  {\normalfont\huge\bfseries}{\thechapter}{16pt}{\huge}
\titlespacing*{\chapter}{0pt}{-10pt}{22pt}

\definecolor{custom-grey}{HTML}{ECECEC}
\newtcbox{\inlinecode}{on line, 
  boxsep=0pt, 
  left=2pt, right=2pt, 
  top=0pt, bottom=0pt, 
  colframe=custom-grey, colback=custom-grey, 
  arc=4pt, boxrule=0pt
}

\begin{document}

\begin{titlepage}
    {\large Mae Boittiaux \par}
    {\huge LINDENMAYER SYSTEM \par}
\end{titlepage}

\tableofcontents

\chapter{Provided Brief}

\section{Aims and Background}

\subsection{Aims}
To implement a graphical demonstration of stochastic Lindenmayer systems and use it to
display organic-like structures.

\subsection{Background}
L-systems or Lindenmayer systems (named after their originator, Aristid Lindenmayer) are
grammar-based mechanisms for producing geometric figures. Like a conventional context
free grammar, an L-system comprises an alphabet and a set of productions. An initial seed
string is used to given as a starting sentential form. Strings are transformed by
applying the productions to all viable elements, and the results strings are interpreted
using a graphics engine.\newline
\newline
Recursive rules may be used to define fractal-like objects, and by attaching
probabilities to individual productions we can produce images which mimic some aspects of
biological systems, including plant growth.\newline
\newline
To succeed in this project, you need to be interested in automata and graphics; and you
must possess good programming skills including the use of GUI's with Java.

\section{Early Deliverables}
\begin{enumerate}
    \item A set of small worked manual examples of non-stochastic L-systems.
    \item A Java program which interprets strings and draws sequences of line segments.
    \item Write a report surveying classes of L-systems.
\end{enumerate}

\section{Final Deliverables}
\begin{enumerate}
    \item Demonstrate a working GUI-based L-system environment.
    \item Extend your system to support stochastic productions and show how they can
          mimic plant growth.
    \item A final report on the project activities, fully acknowledging and citing
          sources.
\end{enumerate}

\section{Suggested Extentions}
\begin{itemize}
    \item Investigate the relationship between L-systems and Semi-Thue systems.
    \item Extend your system to draw 3-dimensional objects.
\end{itemize}

\section{Reading}
\begin{itemize}
    \item The book 'The algorithmic beauty of plants' is available for free.
    \item (\url{http://algorithmicbotany.org/papers/abop/abop.pdf})
    \item Many online resources: start with Wikipedia.
\end{itemize}

\cite{example-cite}

\chapter{Questions}
Here I shall question and hope to answer, the report shall morph from here.

\section{Questions from: Provided Brief}
The following questions were raised from the chapter: `Provided Brief'.

\subsection{What does stochastic mean?}

\subsection{What is a Lindenmayer system?}

\subsection{What are organic like structures?}

\subsection{Are L-Systems and Lindenmayer systems the same thing?}

\subsection{Is Aristid Lindenmayer's intent relevant?}

\subsection{What are grammar based mechanisms?}

\subsection{What is a context free grammar?}

\subsection{What makes a context free grammar conventional?}

\subsection{In this specific context, what is meant by an alphabet and a set of
    productions?}

\subsection{What is the initial seed string?}

\subsection{Can the initial seed string be randomised?}

\subsection{What does sentential mean?}

\subsection{What makes an element viable?}

\subsection{How is the results string interpreted?}

\subsection{What are fractal-like objects?}

\subsection{How are the probabilities determined?}

\subsection{What biological systems are mimiced?}

\subsection{What are automata?}

\subsection{What does non-stochastic mean?}

\subsection{How are L-systems classified?}

\subsection{Why are L-systems classified?}

\subsection{How can L-systems mimic plant growth?}

\subsection{What are Semi-Thue systems?}

\subsection{What is the relationship between L-systems and Semi-Thue systems?}

\subsection{How can the system be extended to draw 3-dimensional objects?}

\clearpage
\printbibliography

\end{document}
