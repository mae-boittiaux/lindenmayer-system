\documentclass[11pt]{report}
\usepackage[utf8]{inputenc}

\usepackage{xcolor}
\usepackage{minted}
\usepackage{caption}
\usepackage{fancyhdr}
\usepackage{lastpage}
\usepackage{titlesec}
\usepackage{hyperref}
\usepackage{tcolorbox}
\usepackage{pmboxdraw}

\usepackage[left=0.75in, right=0.75in, top=0.75in, bottom=1.5in]{geometry}

% bibliography management system using biblatex; backend: biber; citation style: IEEE
\usepackage[backend=biber,style=ieee]{biblatex}
\addbibresource{references.bib}

\renewbibmacro*{finentry}{
    \iffieldundef{annotation}
        {}
        {\par\smallskip\textit{Annotation:} \printfield{annotation}}
    \finentry
}

\author{Hollie-Mae Boittiaux}

\setlength{\headheight}{34pt}
\newcommand{\header}{\vspace{-10pt}\rule{\textwidth}{0.4pt}}
\setlength{\footskip}{33pt}

% pages which start on a chapter
\fancypagestyle{plain}{
    \fancyhf{}
    \fancyhead[L]{\header}
    \fancyfoot[C]{
    \makebox[\textwidth]{
      \rule[3pt]{0.425\textwidth}{0.6pt} 
      \hspace{4pt} 
      Page \thepage\ of \pageref{LastPage} 
      \hspace{4pt} 
      \rule[3pt]{0.425\textwidth}{0.6pt}
    }
  }
    \renewcommand{\headrulewidth}{0pt}
    \renewcommand{\footrulewidth}{0pt}
}

% pages which do not start on a chapter
\pagestyle{fancy}
\fancyhf{}
\fancyhead[L]{\header}
\fancyfoot[C]{
    \makebox[\textwidth]{
      \rule[3pt]{0.425\textwidth}{0.6pt} 
      \hspace{4pt} 
      Page \thepage\ of \pageref{LastPage} 
      \hspace{4pt} 
      \rule[3pt]{0.425\textwidth}{0.6pt}
    }
  }
\renewcommand{\headrulewidth}{0pt}
\renewcommand{\footrulewidth}{0pt}

\titleformat{\chapter}[hang]
  {\normalfont\huge\bfseries}{\thechapter}{16pt}{\huge}
\titlespacing*{\chapter}{0pt}{-10pt}{22pt}

\definecolor{custom-grey}{HTML}{ECECEC}
\newtcbox{\inlinecode}{on line, 
  boxsep=0pt, 
  left=2pt, right=2pt, 
  top=0pt, bottom=0pt, 
  colframe=custom-grey, colback=custom-grey, 
  arc=4pt, boxrule=0pt
}

\begin{document}

\begin{titlepage}
    {\large Mae Boittiaux \par}
    {\huge LINDENMAYER SYSTEM \par}
\end{titlepage}

\tableofcontents

\chapter{Provided Brief}

\section{Aims and Background}

\subsection{Aims}
To implement a graphical demonstration of stochastic Lindenmayer systems and use it to
display organic-like structures.

\subsection{Background}
L-systems or Lindenmayer systems (named after their originator, Aristid Lindenmayer) are
grammar-based mechanisms for producing geometric figures. Like a conventional context
free grammar, an L-system comprises an alphabet and a set of productions. An initial seed
string is used to given as a starting sentential form. Strings are transformed by
applying the productions to all viable elements, and the results strings are interpreted
using a graphics engine.\newline
\newline
Recursive rules may be used to define fractal-like objects, and by attaching
probabilities to individual productions we can produce images which mimic some aspects of
biological systems, including plant growth.\newline
\newline
To succeed in this project, you need to be interested in automata and graphics; and you
must possess good programming skills including the use of GUI's with Java.

\section{Early Deliverables}
\begin{enumerate}
    \item A set of small worked manual examples of non-stochastic L-systems.
    \item A Java program which interprets strings and draws sequences of line segments.
    \item Write a report surveying classes of L-systems.
\end{enumerate}

\section{Final Deliverables}
\begin{enumerate}
    \item Demonstrate a working GUI-based L-system environment.
    \item Extend your system to support stochastic productions and show how they can
          mimic plant growth.
    \item A final report on the project activities, fully acknowledging and citing
          sources.
\end{enumerate}

\section{Suggested Extentions}
\begin{itemize}
    \item Investigate the relationship between L-systems and Semi-Thue systems.
    \item Extend your system to draw 3-dimensional objects.
\end{itemize}

\section{Reading}
\begin{itemize}
    \item The book 'The algorithmic beauty of plants' is available for free.
    \item (\url{http://algorithmicbotany.org/papers/abop/abop.pdf})
    \item Many online resources: start with Wikipedia.
\end{itemize}

\cite{example-cite}

\chapter{Questions}
Here I shall question and hope to answer, the report shall morph from here.

\section{Questions from: Provided Brief}
The following questions were raised from the chapter: `Provided Brief'.

\subsection{What does stochastic mean?}

\subsection{What is a Lindenmayer system?}

\subsection{What are organic like structures?}

\subsection{Are L-Systems and Lindenmayer systems the same thing?}

\subsection{Is Aristid Lindenmayer's intent relevant?}

\subsection{What are grammar based mechanisms?}

\subsection{What is a context free grammar?}

\subsection{What makes a context free grammar conventional?}

\subsection{In this specific context, what is meant by an alphabet and a set of
    productions?}

\subsection{What is the initial seed string?}

\subsection{Can the initial seed string be randomised?}

\subsection{What does sentential mean?}

\subsection{What makes an element viable?}

\subsection{How is the results string interpreted?}

\subsection{What are fractal-like objects?}

\subsection{How are the probabilities determined?}

\subsection{What biological systems are mimiced?}

\subsection{What are automata?}

\subsection{What does non-stochastic mean?}

\subsection{How are L-systems classified?}

\subsection{Why are L-systems classified?}

\subsection{How can L-systems mimic plant growth?}

\subsection{What are Semi-Thue systems?}

\subsection{What is the relationship between L-systems and Semi-Thue systems?}

\subsection{How can the system be extended to draw 3-dimensional objects?}

\chapter{Wikipedia Notes}
The notes taken from the initial reading of:
\url{https://en.wikipedia.org/wiki/L-system}.
\begin{itemize}
    \item an L-system or Lindenmayer system is a parallel rewriting system and a type of
          formal grammar.
    \item an L-system consists of an alphabet of symbols that can be used to make
          strings, a collection of production rules that expand each symbol into some
          larger string of symbols, an initial `axiom' string from which to begin
          construction, and a mechanism for translating the generated strings into
          geometric structures.
    \item L-systems were introduced and developed in 1968 by Aristid Lindenmayer, a
          Hungarian theoretical biologist and botanist at the University of Utrecht.
    \item Lindenmayer used L-systems to describe the behaviour of plant cells and to
          model the growth processes of plant development.
    \item L-systems have also been used to model the morphology of a variety of organisms
          and can be used to generate self-similar fractals.
\end{itemize}

\section{Origins}
\begin{itemize}
    \item as a biologist, Lindenmayer worked with yeast and filamentous fungi and studied
          the growth patterns of various types of bacteria, such as the cyanobacteria
          Anabaena catenula.
    \item originally, the L-systems were devised to provide a formal description of the
          development of such simple multicellular organisms, and to illustrate the
          neighbourhood relationships between plant cells.
    \item later on, this system was extended to describe higher plants and complex
          branching structures.
\end{itemize}

\section{L-System Structure}
\begin{itemize}
    \item the recursive nature of the L-system rules leads to self-similarity and
          thereby, fractal-like forms are easy to describe with an L-system.
    \item plant models and natural-looking organic forms are easy to define, as by
          increasing the recursion level the form slowly `grows' and becomes more
          complex.
    \item Lindenmayer systems are also popular in the generation of artificial life.
    \item L-system grammars are very similar to the semi-Thue grammar (see Chomsky
          hierarchy).
    \item L-systems are now commonly known as parametric L-systems, defined as a tuple:
          $G = (V, \omega, P)$ where:
          \begin{itemize}
              \item $V$ (the alphabet) is a set of symbols containing both elements that
                    can be replaced (variables) and those which cannot be replaced
                    (`constants' or `terminals').
              \item $\omega$ (start, axiom or initiator) is a string of symbols from $V$
                    defining the initial state of the system.
              \item $P$ is a set of production rules or productions defining the way
                    variables can be replaced with combinations of constants and other
                    variables.
                    \begin{itemize}
                        \item a production consists of two strings, the predecessor and
                              the successor.
                        \item for any symbol $A$ which is a member of the set $V$ which
                              does not appear on the left hand side of a production in
                              $P$, the identity production $A -> A$ is assumed; these
                              symbols are called constants or terminals.
                    \end{itemize}
          \end{itemize}
    \item the rules of the L-system grammar are applied iteratively starting from the
          initial state.
    \item as many rules as possible are applied simultaneously, per iteration.
    \item the fact that each iteration employs as many rules as possible differentiates
          an L-system from a formal language generated by a formal grammar, which applies
          only one rule per iteration.
    \item if the production rules were to be applied only one at a time, one would quite
          simply generate a string in a language, and all such sequences of applications
          would produce the language specified by the grammar.
    \item there are some strings in some languages, however, that cannot be generated if
          the grammar is treated as an L-system rather than a language specification.
    \item for example, suppose there is a rule $S->SS$ in a grammar.
    \item if productions are done one at a time, then starting from $S$, we can first get
          $SS$, and then applying the rule again, $SSS$.
    \item however, if all applicable rules are applied at every step, as in an L-system,
          then we cannot get this sentential form.
    \item instead, the first step would give us $SS$, but the second would apply the rule
          twice, giving us $SSSS$.
    \item thus, the set of strings produced by an L-systems from a given grammar is a
          subset of the formal language defined by the grammar, and if we take a language
          to be defined as a set of strings, this means that a given L-system is
          effectively a subset of the formal language defined by the L-system's grammar.
    \item an L-system is context-free if each production rule refers only to an
          individual symbol and not to its neighbours.
    \item context-free L-systems are thus specified by a context-free grammar.
    \item if a rule depends not only on a single symbol but also on its neighbours, it is
          termed a context-sensitive L-system.
    \item if there is exactly one production for each symbol, then the L-system is said
          to be deterministic (a deterministic context-free L-system is popularly called
          a D0L system).
    \item if there are several, and each is chosen with a certain probability during each
          iteration, then it is a stochastic L-system.
    \item using L-systems for generating graphical images requires that the symbols in
          the model refer to elements of a drawing on the computer screen.
    \item for example, the program Fractint uses turtle graphics (similar to those in the
          Logo programming language) to produce screen images.
    \item it interprets each constant in an L-system model as a turtle command.
\end{itemize}

\clearpage
\printbibliography

\end{document}
